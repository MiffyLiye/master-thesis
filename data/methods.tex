\chapter{研究材料与研究方法}
\label{chapter:methods}

我们提出了含有非同步释放的突触模型后,先基于实验数据对模型参数进行拟合,再进行仿真实验。

\section{实验材料}
\label{section:methods:materials}

用于参数拟合的实验数据来自前人实验里 Fast-Spiking 神经元的突触数据,数据采集的更多细节可以参考原始实验文献\cite{Jiang2012}。

\section{分析方法}
\label{section:methods:methods}

我们先对实验数据进行预处理,得到相关的统计信息,然后进行模型的参数拟合。

\subsection{数据预处理}
\label{section:methods:data-preprocessing}

\subsubsection{估计突触电流}
实验记录到的细胞膜电流包括细胞膜的漏电流 $I_\text{leaky}(t)$ 和由于突触信号传递导致的突触电流 $I_\text{syn}(t)$ 。

细胞产生漏电流的离子通道一般是受细胞膜电位调控的,实验条件下细胞膜电位被固定,可以假设漏电流恒定不变。
通过读取突触前细胞发放前一小段时间内突触后神经元的细胞膜电流,通过计算平均值可以得到漏电流 $I_\text{leaky}$ 。
记录的电流减去漏电流 $I_\text{leaky}$ 就能得到突触电流 $I_\text{syn}(t)$ 。
对于抑制性神经元的突触,它所产生的突触电流应该为非正值,但由于噪声的存在,可能有部分时间点 $t$ 时 $I_\text{syn}(t) > 0$。
这时需要对数据进行修正,先取一个极小的电流 $\epsilon > 0$ 作为阈值,对于 $I_\text{syn}(t) > -\epsilon$ 的时间点,将 $I_\text{syn}(t)$ 修改为 $-\epsilon$。
为了不引入太大的偏差, $\epsilon$ 需要足够小。
试验中取了 $\epsilon = 0.2$ pA,相对于IPSC的大小,这是一个非常小的值。

\subsubsection{去卷积方法}


\subsection{数据拟合}
\label{section:methods:data-fitting}
