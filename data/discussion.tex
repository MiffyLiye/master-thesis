\chapter{结果讨论}
\label{chapter:discussion}
本文提出了包含非同步递质释放的突触模型,并进行了单神经元和神经元网络的仿真计算。
讨论也从这三个方面进行。

\section{非同步释放模型结果讨论}
\label{section:discussion:model}

\subsection{非同步释放模型的功能}
\label{section:discussion:model-features}
我们提出的突触模型包含非同步释放和同步释放。
由于模型是基于短时程突触可塑性模型进行扩展,所以模型包含丰富的动力学现象。
我们的模型能与实验数据较好拟合,且能够重现以往实验中观察到的非同步释放的特点。
特别是模型的参数有明显的意义,参数的调节可以对应于实验中的不同处理方式来选择性地改变非同步释放的强度。

前人的实验曾经认为非同步释放和同步释放使用两个不同的活动区域,且两个活动区域距离钙离子通道的距离不一样 \cite{Hefft2005}。但是原来的实验现象也可以用一个活动区域和两种不同的钙离子感受器来解释。两种钙离子感受器具有不同的结合速率、分解速率、和亲和性。
在我们的模型中,我们假设差异是来自两种钙离子感受器。
这里有一定的简化,有一个包含更多生物细节的的模型只用一种钙离子感受器,但详细模拟了钙离子浓度的变化,也重复出了类似的现象 \cite{Pan2009}。
然而基于简化模型的考虑,假设有两种钙离子感受器竞争同一堆囊泡资源更加简单而且可以重复实验现象。

另一个模型基于分子机制描述非同步释放现象 \cite{Nadkarni2010}。
描述海马区域的非同步释放的分子细节使得这个模型计算量较大,不适合进行神经元网络的仿真计算。
要特别注意到,由于突出的数量随着神经元数目 $n$ 的增加数量会近似为 $O(n^2)$ ,所以突触的计算量会在大型网络仿真计算中占主要部分。

我们提出的模型计算高效同时具有实验中发现的一些主要特点。
它适合应用于神经元网络仿真计算,来研究非同步释放在网络层面的影响。
通过调节模型的参数能够模拟各种类型的非同步释放。

\section{非同步释放模型的局限}
\label{section:discussion:limitations}

\subsection{实验数据预处理的局限}
\label{section:discussion:preprocessing}
一些数据处理时的假设可能在实际情况中不成立,这样会影响用于拟合的数据的准确性。

例如预处理中假设漏电流为常数,但是实际情况是它会随细胞膜电位和神经元内部状态的变化而变化。
还有,反转电位 $E_\text{GABA}$ 被假设为常数,但是在高频率刺激下,细胞可能无法保持在平衡状态 \cite{Staley1995},因此导致 $E_\text{GABA}$ 偏移。

在估计模型参数的时候,突触后的递质接收器的动力学被假设为单指数递减型,这个限制了分析方法应用于某些神经元突触。
首先突触的饱和作用就被忽略了。
虽然对于更复杂的突触类型,释放率统计会变得更困难,但是突触释放模型的核心不受影响,因为它只模拟释放过程,而不受递质接收器的动力学影响。

综上,一些假设限制了估计的释放率的准确度,进而影响参数拟合的准确度。
虽然量化分析的结果可能会有些变化,但应该不会影响非同步释放的定性描述。

\section{单突触仿真结果讨论}
\label{section:discussion:synapse-simulation}
单突触仿真的对非同步释放强度的调控与实验现象对应,因此模型揭示了不同实验调控方法提高非同步释放强度的一些差异(见第\ref{section:result:modulate-with-tau-ar}节和第\ref{section:result:modulate-with-u-ar}节)。

此外第\ref{section:result:modulate-x-0}节的结果显示了实验中尚未研究过的现象。
量子相对大小$wx_0$的变化会改变 IPSC 的平滑度。
如果采用 IPSC 斜率来探测非同步释放事件,可能会加大阈值挑选的难度。
根据模型,如果要用 IPSC 的斜率来探测非同步释放事件,那么应该先推断出量子大小 $wx_0$,然后根据量子大小设置阈值。

\section{网络仿真结果讨论}
\label{section:discussion:network-simulation}
前人实验认为非同步释放会使网络发放时间精确性下降 \cite{Manseau2010},进而使网络去同步。
仿真计算发现非同步释放造成的发放精度下降依赖于突触连接密度,连接较密集时影响较小。
实验中没有观察到这种现象,可能是因为实验材料中回路较小,突触连接稀疏。

此外由于竞争囊泡资源,非同步释放的增强会导致同步释放的减弱。
而同步释放使网络同步的作用比非同步释放强,即每个周期内不同神经元的相位差改变大。
结果中的对相位反应曲线的分析,解释了前人网络仿真实验中发现的较小的 $\tau_\text{GABA}$ 能提高网络同步的能力 \cite{Bartos2002}。
因此,非同步释放增强可能使网络同步行为对噪声的抵抗力降低。
所以可以认为非同步释放增强,可以通过影响同步释放间接影响网络同步性。

此外,以前一些实验中用电导 $g$ 来衡量抑制性水平 \cite{Wang1996,Voegtlin2007},而网络仿真的结果显示,相同的平均电导 $E(g)$ 也可能有不同的抑制性水平(见第\ref{section:result:network-self-inhibition}节)。选择发放率作为抑制性指标能够揭示被忽视的作用机制。

\section{总结}
我们展示了包含同步释放和非同步释放的化学突触模型。
模型基于大型网络仿真计算中常用的短时程突触可塑性模型,计算效率较高,适合未来的模型研究。
本文进行了简单的网络模拟,得出了一些结果。
使得对以往实验的理解又深入了一步。
希望这一模型能够促进非同步释放研究,增加对它在神经系统中的计算特性和神经功能的理解。
