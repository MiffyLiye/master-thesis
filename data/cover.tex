
%%% Local Variables:
%%% mode: latex
%%% TeX-master: t
%%% End:
\secretlevel{绝密} \secretyear{2100}

\ctitle{非同步神经递质释放的模型和仿真计算研究}
% 根据自己的情况选,不用这样复杂
\makeatletter
\ifthu@bachelor\relax\else
  \ifthu@doctor
    \cdegree{理学博士}
  \else
    \ifthu@master
      \cdegree{理学硕士}
    \fi
  \fi
\fi
\makeatother


\cdepartment[心理学]{脑与认知科学研究院}
\cmajor{认知神经科学}
\cauthor{王韬} 
\csupervisor{吴思教授}
% 如果没有副指导老师或者联合指导老师,把下面两行相应的删除即可。
%\cassosupervisor{Malte J Rasch教授}
%\ccosupervisor{王大辉教授}
% 日期自动生成,如果你要自己写就改这个cdate
%\cdate{\CJKdigits{\the\year}年\CJKnumber{\the\month}月}

% 博士后部分
% \cfirstdiscipline{计算机科学与技术}
% \cseconddiscipline{系统结构}
% \postdoctordate{2009年7月——2011年7月}

\etitle{Model and Computer Simulation of the Asynchronous Neurotransmitter Release} 
% 这块比较复杂,需要分情况讨论:
% 1. 学术型硕士
%    \edegree:必须为Master of Arts或Master of Science(注意大小写)
%              “哲学、文学、历史学、法学、教育学、艺术学门类,公共管理学科
%               填写Master of Arts,其它填写Master of Science”
%    \emajor:“获得一级学科授权的学科填写一级学科名称,其它填写二级学科名称”
% 2. 专业型硕士
%    \edegree:“填写专业学位英文名称全称”
%    \emajor:“工程硕士填写工程领域,其它专业学位不填写此项”
% 3. 学术型博士
%    \edegree:Doctor of Philosophy(注意大小写)
%    \emajor:“获得一级学科授权的学科填写一级学科名称,其它填写二级学科名称”
% 4. 专业型博士
%    \edegree:“填写专业学位英文名称全称”
%    \emajor:不填写此项
\edegree{Master of Science} 
\emajor{Cognitive Neuroscience} 
\eauthor{Wang Tao} 
\esupervisor{Professor Wu Si} 
%\eassosupervisor{Associate Professor Malte J Rasch} 
% 这个日期也会自动生成,你要改么?
% \edate{December, 2005}

% 定义中英文摘要和关键字
\begin{cabstract}
神经元通过突触传递信息。
动作电位引起轴突末梢的神经递质释放。
通常,神经递质的释放与动作电位的到达在时间上紧密锁定,因此被称为同步释放。
然而,神经递质的释放可以高度随机,而且在动作电位停止发放后很久还有递质释放,这是非同步释放。
非同步释放会随着组织和神经元类型的变化而变化。
最近实验发现快速发放神经元中也有。
要注意,在人类癫痫病人的组织中发现非同步释放增强,意味着它可能对产生病态神经活动有重要作用。
现有的用于通过神经元网络仿真计算来研究神经活动的模型几乎都只考虑同步释放而忽略了非同步释放。
这里,我们提出一种包含非同步释放的唯相模型,它一方面具有非同步释放的几个基本的特点,另一方面比较简单因而适用于大规模网络仿真。
我们提出的模型基于著名的短时程突触可塑性模型,增加了随机释放的部分来模拟非同步释放。
我们使用人类癫痫组织中的抑制性快速发放神经元的实验数据来拟合模型参数,进而展示出我们的模型能够重现真实非同步递质释放的特性。
我们也在同步化发放的神经元网络上进行了一些与前人实验相关的仿真计算,分析了非同步释放的影响以及作用机制。
\end{cabstract}

\ckeywords{突触模型,非同步释放,短时程突触可塑性,随机释放}

\begin{eabstract} 
Neurons communicate with each other via synapses.  Action
potentials cause release of neurotransmitters at the axon
terminal. Typically, this neurotransmitter release is tightly time-locked to
the arrival of an action potential and is thus called synchronous
release. However, neurotransmitter release is stochastic and the
rate of release of small quanta of neurotransmitters can be
considerably elevated even long after the ceasing of spiking
activity, leading to asynchronous release of neurotransmitters. 
Such asynchronous release varies for tissue and neuron
types and has been shown recently to be pronounced in fast-spiking
neurons.  Notably, it was found that asynchronous release is
enhanced in human epileptic tissue implicating a possibly important
role in generating abnormal neural activity. Current neural
network models for simulating and studying neural activity virtually only
consider synchronous release and ignore asynchronous transmitter
release.  Here, we develop a phenomenological model for asynchronous
neurotransmitter release, which, on one hand, captures the
fundamental features of the asynchronous release process, and, on
the other hand, is simple enough to be incorporated in large-size
network simulations.  Our proposed model is based on the well-known
equations for short-term dynamical synaptic interactions and
includes an additional stochastic term for modelling asynchronous
release. We use experimental data obtained from inhibitory
fast-spiking synapses of human epileptic tissue to fit the model
parameters, and demonstrate that our model reproduces the
characteristics of realistic asynchronous transmitter release. 
We also did some network simulations in the synchronized fast-spiking neural network relevant to previous experiments, and analyzed the mechanisms.
\end{eabstract}

\ekeywords{synapse model,asynchronous release,short-term synaptic plasticity,stochastic release}
