\chapter{计算模型}
\label{chapter:model}

\section{神经元模型}
\label{section:model:neuron-model}

常见的神经元模型包括 Hodgkin-Huxley 模型和 Integrate-and-Fire 模型。
前者可以模拟细胞膜上各种离子通道,通过调整参数,可以模拟出大量的特性。
后者不模拟离子通道的细节,只模拟细胞膜电位的变化,计算更高效。
介于两只之间的有 Izhikevich 模型,模型的部分参数失去了直观的物理意义,但通过调整参数,可以模拟出很多典型的神经元发放特性,同时计算也比 Hodgkin-Huxley 模型高效很多 \cite{Izhikevich2003,Izhikevich2004}。

\subsection{Izhikevich 模型}
Izhikevich 模型的动力学方程如下
\begin{align}
C \frac{\dd{v}}{\dd{t}} &= k \left( v-v_\text{r} \right) \left( v-v_\text{t} \right) - u + I \label{equation:izhikevich-membrane} \\
\frac{\dd{u}}{\dd{t}} &= a \left\{ b \left( v-v_\text{r} \right) - u \right\} \label{equation:izhikevich-recovery-current} \\
\text{当} v \geq v_{\text{peak}} \text{时,} & v \leftarrow c, u \leftarrow u + d \label{equation:izhikevich-spike}
\end{align}
其中 $v$ 是细胞膜电位, $u$ 是恢复电流, $C$ 是细胞膜电容, $v_\text{r}$ 是细胞膜静息电位, $v_\text{t}$ 是细胞膜自主发放电位。
$a$, $b$, $c$, $d$,和 $k$是其他参数, $I$ 是外部输入的电流。

在模拟 Fast-Spiking 神经元时,对模型中的式\ref{equation:izhikevich-recovery-current}进行一些改动,
\begin{align}
\frac{\dd{u}}{\dd{t}} &= 0.2 \left\{ U\left( v \right) - u \right\} \label{equation:izhikevich-fast-spiking} \\
U\left( v \right) &= \begin{cases} 0, &\mbox{当} v < v_\text{b} \mbox{时} \\ 0.025, &\mbox{当} v \geq v_\text{b} \mbox{时} \end{cases}
\end{align}
各项参数为
\begin{align*}
C &= 20\ \text{pF}, \\
k &= 1\ \text{pA} \cdot \text{mV$^{-2}$}, \\
v_\text{r} &= -55\ \text{mV}, \\
v_\text{t} &= -40\ \text{mV}, \\
v_\text{b} &= -55\ \text{mV}, \\
v_\text{peak} &= 25\ \text{mV}, \\
c &= -45\ \text{mV}.
\end{align*}

\section{突触模型}
\label{section:model:synapse-model}
流行的描述突触动态性的模型有 Misha Tsodyks 和 Henry Markram 提出的短时程突触可塑性(Short-Term Synaptic Plasticity)模型\cite{Markram1996,Tsodyks1997}。
最新版本的模型如下\cite{Tsodyks2013}

\begin{align}
\frac{\dd{u\left(t\right)}}{\dd{t}} &=  -\frac{u\left(t\right)}{\tau_f}+U\left[1-u\left(t\right)\right]\sum_m \delta\left(t-t_m\right), \label{equation:stp-facilitation} \\
\frac{\dd{x\left(t\right)}}{\dd{t}} &=
\frac{X_F-x\left(t\right)}{\tau_d} - q(t),
\label{equation:stp-depression} \\
q(t) & = u\left(t_+\right)x\left(t_-\right) \sum_m \delta\left(t-t_m\right),
\label{equation:stp-release}
\end{align}
其中 $u$ 代表细胞膜内钙离子残余造成的递质释放概率提升, $x$ 代表剩余的可释放递质的量, $q$ 代表递质的释放速率。 
$X_F$ 是静息状态下突触的可释放递质的总量。 
$\tau_f$, $U$, $\tau_d$ 是其他参数。不同类型的神经元有不同的参数 \cite{Silberberg2005}。

释放出的递质 $q$ 与突触后神经元上的递质接收器结合,产生突触后神经元细胞膜电流等一系列变化。
这部分也可以用模型描述\cite{Destexhe1994}。
\begin{align}
I_\text{syn}(t) &= g(t)\left(v(t) - E_\text{syn}\right), \label{equation:synaptic-current-general} \\
\frac{\dd{g(t)}}{\dd{t}} &= -\frac{g(t)}{\tau_\text{syn}} + wq(t - t_\text{delay}), \label{equation:receptor-conductance-general}
\end{align}
其中 $v$ 是突触后细胞的膜电位, $g$ 是离子通道的电导, $E_\text{syn}$ 是离子的反转电位, $w$ 是突触权重, $t_\text{delay}$ 是由于递质传递造成的延迟。

\subsection{非同步递质释放模型}
\label{section:model:asynchronous-release}
基于短时程突触可塑性模型,我们提出非同步递质释放模型,在原模型的基础上,增加非同步释放机制。
基于囊泡上存在两类不同特性的钙离子感受器,我们将式\ref{equation:stp-facilitation}分成两个形式一致,但具有不同参数的过程。
\begin{align}
\frac{\dd{u_\text{sr}\left(t\right)}}{\dd{t}} &= -\frac{u_\text{sr}\left(t\right)}{\tau_\text{sr}}+U_\text{sr}\left[1-u_\text{sr}\left(t\right)\right]\sum_m \delta\left(t-t_m\right), \label{equation:sr-facilitation} \\
\frac{\dd{u_\text{ar}\left(t\right)}}{\dd{t}} &= -\frac{u_\text{ar}\left(t\right)}{\tau_\text{ar}}+U_\text{ar}\left[1-u_\text{ar}\left(t\right)\right]\sum_m \delta\left(t-t_m\right), \label{equation:ar-facilitation}
\end{align}
其中 $u_\text{sr}$ 和 $u_\text{ar}$ 是由于钙离子参与造成的造成的同步释放与非同步释放的不同的激活程度。
$U_\text{sr}$ 和 $U_\text{ar}$ 控制每个动作电位之后同步释放与非同步释放激活程度的提升大小。
$\tau_\text{sr}$ 和 $\tau_\text{ar}$ 控制钙离子感受器与钙离子分离造成同步释放与非同步释放激活程度下降的速度。

由于有实验支持同步释放和非同步释放竞争同一个递质资源池\cite{Otsu2004,Wen2010},所以使用单个变量 $x$ 来代表剩余的递质量,依旧沿用式\ref{equation:stp-depression}。

沿用原模型里的描述,只有动作电位到达后的短暂时刻有同步释放,同步释放的释放率为
\begin{equation}
q_\text{sr}(t) = u_\text{sr}(t_+)x(t_-)\sum_m \delta(t-t_m). \label{equation:sr-release}
\end{equation}

非同步释放则是高度随机的事件。
随机性使用二项过程模拟 \cite{DELCASTILLO1954}。
随机变量 $n$ 服从二项分布, $n\sim{\cal{B}}\left(N,p\right)$, 则它的概率分布函数为 $P(n=k|N,p) = \binom{N}{k} p^k (1-p)^{N-k}$, 其中 $k\in\{0,\ldots,N\}$, 且 $p$ 是单个事件发生的概率。



\section{网络模型}
\label{section:model:network-model}

\section{简化模型}
\label{section:model:simplified-model}
