%%% Local Variables:
%%% mode: latex
%%% TeX-master: t
%%% End:

%\documentclass[bachelor,nofonts]{thuthesis}
\documentclass[master,winfonts]{thuthesis}
%\documentclass[doctor]{thuthesis}
% \documentclass[%
%   bachelor|master|doctor|postdoctor, % mandatory option
%   winfonts|nofonts|adobefonts, % mandatory only for bachelor and Linuxer
%   secret,
%   openany|openright,
%   arialtoc,arialtitle]{thuthesis}
% 当使用 XeLaTeX 编译时,本科生、Linux 用户需要加上 nofonts 选项;
% 当使用 PDFLaTeX 编译时,adobefonts 选项等效于 winfonts 选项(缺省选项)。

% 所有其它可能用到的包都统一放到这里了,可以根据自己的实际添加或者删除。
\usepackage{thutils}

% 你可以在这里修改配置文件中的定义,导言区可以使用中文。
% \def\myname{薛瑞尼}

\usepackage{amsmath}
\usepackage{amssymb}
\usepackage{bm}
\usepackage{metalogo}

\newcommand{\dd}[1]{\,\mathrm{d}{#1}}
\DeclareMathOperator{\argmax}{arg\,max}
\DeclareMathOperator*{\Argmax}{arg\,max}
\DeclareMathOperator{\argmin}{arg\,min}
\DeclareMathOperator*{\Argmin}{arg\,min}

\begin{document}

% 定义所有的eps文件在 figures 子目录下
\graphicspath{{figures/}}


%%% 封面部分
\frontmatter

%%% Local Variables:
%%% mode: latex
%%% TeX-master: t
%%% End:
\secretlevel{绝密} \secretyear{2100}

\ctitle{非同步神经递质释放的模型和仿真计算研究}
% 根据自己的情况选,不用这样复杂
\makeatletter
\ifthu@bachelor\relax\else
  \ifthu@doctor
    \cdegree{理学博士}
  \else
    \ifthu@master
      \cdegree{理学硕士}
    \fi
  \fi
\fi
\makeatother


\cdepartment[脑科院]{脑与认知科学研究院}
\cmajor{认知神经科学}
\cauthor{王韬} 
\csupervisor{吴思 ~ 教授}
% 如果没有副指导老师或者联合指导老师,把下面两行相应的删除即可。
%\cassosupervisor{Malte J Rasch教授}
%\ccosupervisor{王大辉教授}
% 日期自动生成,如果你要自己写就改这个cdate
%\cdate{\CJKdigits{\the\year}年\CJKnumber{\the\month}月}

% 博士后部分
% \cfirstdiscipline{计算机科学与技术}
% \cseconddiscipline{系统结构}
% \postdoctordate{2009年7月——2011年7月}

\etitle{Model and Computer Simulation of the Asynchronous Neurotransmitter Release} 
% 这块比较复杂,需要分情况讨论:
% 1. 学术型硕士
%    \edegree:必须为Master of Arts或Master of Science(注意大小写)
%              “哲学、文学、历史学、法学、教育学、艺术学门类,公共管理学科
%               填写Master of Arts,其它填写Master of Science”
%    \emajor:“获得一级学科授权的学科填写一级学科名称,其它填写二级学科名称”
% 2. 专业型硕士
%    \edegree:“填写专业学位英文名称全称”
%    \emajor:“工程硕士填写工程领域,其它专业学位不填写此项”
% 3. 学术型博士
%    \edegree:Doctor of Philosophy(注意大小写)
%    \emajor:“获得一级学科授权的学科填写一级学科名称,其它填写二级学科名称”
% 4. 专业型博士
%    \edegree:“填写专业学位英文名称全称”
%    \emajor:不填写此项
\edegree{Master of Science} 
\emajor{Cognitive Neuroscience} 
\eauthor{Wang Tao} 
\esupervisor{Professor Wu Si} 
%\eassosupervisor{Associate Professor Malte J Rasch} 
% 这个日期也会自动生成,你要改么?
% \edate{December, 2005}

% 定义中英文摘要和关键字
\begin{cabstract}
神经元通过突触传递信息。
动作电位引起轴突末梢的神经递质释放。
通常,神经递质的释放与动作电位的到达在时间上紧密锁定,因此被称为同步释放。
然而,神经递质的释放可以高度随机,而且在动作电位停止发放后很久还有递质释放,这是非同步释放。
非同步释放会随着组织和神经元类型的变化而变化。
最近实验发现快速发放神经元中也有。
要注意,在人类癫痫病人的组织中发现非同步释放增强,意味着它可能对产生病态神经活动有重要作用。
现有的用于通过神经元网络仿真计算来研究神经活动的模型几乎都只考虑同步释放而忽略了非同步释放。
这里,我们提出一种包含非同步释放的唯相模型,它一方面具有非同步释放的几个基本的特点,另一方面比较简单因而适用于大规模网络仿真。
我们提出的模型基于著名的短时程突触可塑性模型,增加了随机释放的部分来模拟非同步释放。
我们使用人类癫痫组织中的抑制性快速发放神经元的实验数据来拟合模型参数,进而展示出我们的模型能够重现真实非同步递质释放的特性。
我们也在同步化发放的神经元网络上进行了一些与前人实验相关的仿真计算,分析了非同步释放的影响以及作用机制。
\end{cabstract}

\ckeywords{突触模型,非同步释放,短时程突触可塑性,随机释放}

\begin{eabstract} 
Neurons communicate with each other via synapses.  Action
potentials cause release of neurotransmitters at the axon
terminal. Typically, this neurotransmitter release is tightly time-locked to
the arrival of an action potential and is thus called synchronous
release. However, neurotransmitter release is stochastic and the
rate of release of small quanta of neurotransmitters can be
considerably elevated even long after the ceasing of spiking
activity, leading to asynchronous release of neurotransmitters. 
Such asynchronous release varies for tissue and neuron
types and has been shown recently to be pronounced in fast-spiking
neurons.  Notably, it was found that asynchronous release is
enhanced in human epileptic tissue implicating a possibly important
role in generating abnormal neural activity. Current neural
network models for simulating and studying neural activity virtually only
consider synchronous release and ignore asynchronous transmitter
release.  Here, we develop a phenomenological model for asynchronous
neurotransmitter release, which, on one hand, captures the
fundamental features of the asynchronous release process, and, on
the other hand, is simple enough to be incorporated in large-size
network simulations.  Our proposed model is based on the well-known
equations for short-term dynamical synaptic interactions and
includes an additional stochastic term for modelling asynchronous
release. We use experimental data obtained from inhibitory
fast-spiking synapses of human epileptic tissue to fit the model
parameters, and demonstrate that our model reproduces the
characteristics of realistic asynchronous transmitter release. 
We also did some network simulations in the synchronized fast-spiking neural network relevant to previous experiments, and analyzed the mechanisms.
\end{eabstract}

\ekeywords{synapse model, asynchronous release, short-term synaptic plasticity, stochastic release}

% 设置 PDF 文档的作者、主题等属性
\makeatletter
\thu@setup@pdfinfo
\makeatother
\makecover % find details in thuthesis.cls \newcommand{\makecover}

% 目录
\tableofcontents

% 符号对照表
\begin{denotation}

\item[Short-Term Synaptic Plasticity] 短时程突触可塑性
\item[Fast-Spiking Neuron] 快速发放神经元
\item[Pyramidal Neuron] 金字塔形神经元
\item[Synapse] 突触
\item[Receptor] 接收器
\item[Decay Time Constant] 衰减常数
\item[cholecystokinin] CCK
\item[parvalbumin] PV
\item[in-phase] 同相位
\item[anti-phase] 反相位
\item[phase response curve] 相位反应曲线
\item[association] 结合
\item[dissociation] 分解
\item[cooperativity] 亲和性
\end{denotation}



%%% 正文部分
\mainmatter
%\include{data/chap01}
%\include{data/chap02}

\chapter{背景}
\label{chapter:background}

\section{引言}
\label{section:background-introduction}

\section{突触模型}
\label{section:background-synapse-model}

\section{论文概要}
\label{section:background-outline}
%\chapter{回顾}
\label{chapter:review}

\section{非同步递质释放的生物机制}
\label{section:review:vesicle-release-mechanism}

\section{神经元突触的计算模型}
\label{section:review:neural-computation-model}

\section{总结}
\label{section:review:summary}
\chapter{计算模型}
\label{chapter:model}

\section{神经元模型}
\label{section:model:neuron-model}

常见的神经元模型包括 Hodgkin-Huxley 模型和 Integrate-and-Fire 模型。
前者可以模拟细胞膜上各种离子通道,通过调整参数,可以模拟出大量的特性。
后者不模拟离子通道的细节,只模拟细胞膜电位的变化,计算更高效。
介于两只之间的有 Izhikevich 模型,模型的部分参数失去了直观的物理意义,但通过调整参数,可以模拟出很多典型的神经元发放特性,同时计算也比 Hodgkin-Huxley 模型高效很多 \cite{Izhikevich2003,Izhikevich2004}。

\subsection{Izhikevich 模型}
Izhikevich 模型的动力学方程如下
\begin{align}
C \frac{\dd{v}}{\dd{t}} &= k \left( v-v_\text{r} \right) \left( v-v_\text{t} \right) - u + I \label{equation:izhikevich-membrane} \\
\frac{\dd{u}}{\dd{t}} &= a \left\{ b \left( v-v_\text{r} \right) - u \right\} \label{equation:izhikevich-recovery-current} \\
\text{当} v \geq v_{\text{peak}} \text{时,} & v \leftarrow c, u \leftarrow u + d \label{equation:izhikevich-spike}
\end{align}
其中 $v$ 是细胞膜电位, $u$ 是恢复电流, $C$ 是细胞膜电容, $v_\text{r}$ 是细胞膜静息电位, $v_\text{t}$ 是细胞膜自主发放电位。
$a$, $b$, $c$, $d$,和 $k$是其他参数, $I$ 是外部输入的电流。

在模拟 Fast-Spiking 神经元时,对模型中的式\ref{equation:izhikevich-recovery-current}进行一些改动,
\begin{align}
\frac{\dd{u}}{\dd{t}} &= 0.2 \left\{ U\left( v \right) - u \right\} \label{equation:izhikevich-fast-spiking} \\
U\left( v \right) &= \begin{cases} 0, &\mbox{当} v < v_\text{b} \mbox{时} \\ 0.025, &\mbox{当} v \geq v_\text{b} \mbox{时} \end{cases}
\end{align}
各项参数为
\begin{align*}
C &= 20\ \text{pF}, \\
k &= 1\ \text{pA} \cdot \text{mV$^{-2}$}, \\
v_\text{r} &= -55\ \text{mV}, \\
v_\text{t} &= -40\ \text{mV}, \\
v_\text{b} &= -55\ \text{mV}, \\
v_\text{peak} &= 25\ \text{mV}, \\
c &= -45\ \text{mV}.
\end{align*}

\section{突触模型}
\label{section:model:synapse-model}
流行的描述突触动态性的模型有 Misha Tsodyks 和 Henry Markram 提出的短时程突触可塑性(Short-Term Synaptic Plasticity)模型\cite{Markram1996,Tsodyks1997}。
最新版本的模型如下\cite{Tsodyks2013}

\begin{align}
\frac{\dd{u\left(t\right)}}{\dd{t}} &=  -\frac{u\left(t\right)}{\tau_f}+U\left[1-u\left(t\right)\right]\sum_m \delta\left(t-t_m\right), \label{equation:stp-facilitation} \\
\frac{\dd{x\left(t\right)}}{\dd{t}} &=
\frac{X_F-x\left(t\right)}{\tau_d} - q(t),
\label{equation:stp-depression} \\
q(t) & = u\left(t_+\right)x\left(t_-\right) \sum_m \delta\left(t-t_m\right),
\label{equation:stp-release}
\end{align}
其中 $u$ 代表细胞膜内钙离子残余造成的递质释放概率提升, $x$ 代表剩余的可释放递质的量, $q$ 代表递质的释放速率。 
$X_F$ 是静息状态下突触的可释放递质的总量。 
$\tau_f$, $U$, $\tau_d$ 是其他参数。不同类型的神经元有不同的参数 \cite{Silberberg2005}。

释放出的递质 $q$ 与突触后神经元上的递质接收器结合,产生突触后神经元细胞膜电流等一系列变化。
这部分也可以用模型描述\cite{Destexhe1994}。
\begin{align}
I_\text{syn}(t) &= g(t)\left(v(t) - E_\text{syn}\right), \label{equation:synaptic-current-general} \\
\frac{\dd{g(t)}}{\dd{t}} &= -\frac{g(t)}{\tau_\text{syn}} + wq(t - t_\text{delay}), \label{equation:receptor-conductance-general}
\end{align}
其中 $v$ 是突触后细胞的膜电位, $g$ 是离子通道的电导, $E_\text{syn}$ 是离子的反转电位, $w$ 是突触权重, $t_\text{delay}$ 是由于递质传递造成的延迟。

\subsection{非同步递质释放模型}
\label{section:model:asynchronous-release}
基于短时程突触可塑性模型,我们提出非同步递质释放模型,在原模型的基础上,增加非同步释放机制。
基于囊泡上存在两类不同特性的钙离子感受器,我们将式\ref{equation:stp-facilitation}分成两个形式一致,但具有不同参数的过程。
\begin{align}
\frac{\dd{u_\text{sr}\left(t\right)}}{\dd{t}} &= -\frac{u_\text{sr}\left(t\right)}{\tau_\text{sr}}+U_\text{sr}\left[1-u_\text{sr}\left(t\right)\right]\sum_m \delta\left(t-t_m\right), \label{equation:sr-facilitation} \\
\frac{\dd{u_\text{ar}\left(t\right)}}{\dd{t}} &= -\frac{u_\text{ar}\left(t\right)}{\tau_\text{ar}}+U_\text{ar}\left[1-u_\text{ar}\left(t\right)\right]\sum_m \delta\left(t-t_m\right), \label{equation:ar-facilitation}
\end{align}
其中 $u_\text{sr}$ 和 $u_\text{ar}$ 是由于钙离子参与造成的造成的同步释放与非同步释放的不同的激活程度。
$U_\text{sr}$ 和 $U_\text{ar}$ 控制每个动作电位之后同步释放与非同步释放激活程度的提升大小。
$\tau_\text{sr}$ 和 $\tau_\text{ar}$ 控制钙离子感受器与钙离子分离造成同步释放与非同步释放激活程度下降的速度。

由于有实验支持同步释放和非同步释放竞争同一个递质资源池\cite{Otsu2004,Wen2010},所以使用单个变量 $x$ 来代表剩余的递质量,依旧沿用式\ref{equation:stp-depression}。

沿用原模型里的描述,只有动作电位到达后的短暂时刻有同步释放,同步释放的释放率为
\begin{equation}
q_\text{sr}(t) = u_\text{sr}(t_+)x(t_-)\sum_m \delta(t-t_m). \label{equation:sr-release}
\end{equation}

非同步释放则是高度随机的事件。
随机性使用二项过程模拟 \cite{DELCASTILLO1954}。
随机变量 $n$ 服从二项分布, $n\sim{\cal{B}}\left(N,p\right)$, 则它的概率分布函数为 $P(n=k|N,p) = \binom{N}{k} p^k (1-p)^{N-k}$, 其中 $k\in\{0,\ldots,N\}$, 且 $p$ 是单个事件发生的概率。



\section{网络模型}
\label{section:model:network-model}

\section{简化模型}
\label{section:model:simplified-model}

\chapter{研究材料与研究方法}
\label{chapter:methods}

我们提出了含有非同步释放的突触模型后,先基于实验数据对模型参数进行拟合,再进行仿真实验。

\section{实验材料}
\label{section:methods:materials}

用于参数拟合的实验数据来自前人实验里 Fast-Spiking 神经元的突触数据,数据采集的更多细节可以参考原始实验文献\cite{Jiang2012}。

\section{分析方法}
\label{section:methods:methods}

我们先对实验数据进行预处理,得到相关的统计信息,然后进行模型的参数拟合。

\subsection{数据预处理}
\label{section:methods:data-preprocessing}

\subsubsection{估计突触电流}
实验记录到的细胞膜电流包括细胞膜的漏电流 $I_\text{leaky}(t)$ 和由于突触信号传递导致的突触电流 $I_\text{syn}(t)$ 。

细胞产生漏电流的离子通道一般是受细胞膜电位调控的,实验条件下细胞膜电位被固定,可以假设漏电流恒定不变。
通过读取突触前细胞发放前一小段时间内突触后神经元的细胞膜电流,通过计算平均值可以得到漏电流 $I_\text{leaky}$ 。
记录的电流减去漏电流 $I_\text{leaky}$ 就能得到突触电流 $I_\text{syn}(t)$ 。
对于抑制性神经元的突触,它所产生的突触电流应该为非正值,但由于噪声的存在,可能有部分时间点 $t$ 时 $I_\text{syn}(t) > 0$。
这时需要对数据进行修正,先取一个极小的电流 $\epsilon > 0$ 作为阈值,对于 $I_\text{syn}(t) > -\epsilon$ 的时间点,将 $I_\text{syn}(t)$ 修改为 $-\epsilon$。
为了不引入太大的偏差, $\epsilon$ 需要足够小。
试验中取了 $\epsilon = 0.2$ pA,相对于IPSC的大小,这是一个非常小的值。

\subsubsection{去卷积方法}


\subsection{数据拟合}
\label{section:methods:data-fitting}

\chapter{仿真计算结果}
\label{chapter:results}
\chapter{结果讨论}
\label{chapter:discussion}
本文提出了包含非同步递质释放的突触模型,并进行了单神经元和神经元网络的仿真计算。
讨论也从这三个方面进行。

\section{非同步释放模型结果讨论}
\label{section:discussion:model}

\subsection{非同步释放模型的功能}
\label{section:discussion:model-features}
我们提出的突触模型包含非同步释放和同步释放。
由于模型是基于短时程突触可塑性模型进行扩展,所以模型包含丰富的动力学现象。
我们的模型能与实验数据较好拟合,且能够重现以往实验中观察到的非同步释放的特点。
特别是模型的参数有明显的意义,参数的调节可以对应于实验中的不同处理方式来选择性地改变非同步释放的强度。

前人的实验曾经认为非同步释放和同步释放使用两个不同的活动区域,且两个活动区域距离钙离子通道的距离不一样 \cite{Hefft2005}。但是原来的实验现象也可以用一个活动区域和两种不同的钙离子感受器来解释。两种钙离子感受器具有不同的结合速率、分解速率、和亲和性。
在我们的模型中,我们假设差异是来自两种钙离子感受器。
这里有一定的简化,有一个包含更多生物细节的的模型只用一种钙离子感受器,但详细模拟了钙离子浓度的变化,也重复出了类似的现象 \cite{Pan2009}。
然而基于简化模型的考虑,假设有两种钙离子感受器竞争同一堆囊泡资源更加简单而且可以重复实验现象。

另一个模型基于分子机制描述非同步释放现象 \cite{Nadkarni2010}。
描述海马区域的非同步释放的分子细节使得这个模型计算量较大,不适合进行神经元网络的仿真计算。
要特别注意到,由于突出的数量随着神经元数目 $n$ 的增加数量会近似为 $O(n^2)$ ,所以突触的计算量会在大型网络仿真计算中占主要部分。

我们提出的模型计算高效同时具有实验中发现的一些主要特点。
它适合应用于神经元网络仿真计算,来研究非同步释放在网络层面的影响。
通过调节模型的参数能够模拟各种类型的非同步释放。

\section{非同步释放模型的局限}
\label{section:discussion:limitations}

\subsection{实验数据预处理的局限}
\label{section:discussion:preprocessing}
一些数据处理时的假设可能在实际情况中不成立,这样会影响用于拟合的数据的准确性。

例如预处理中假设漏电流为常数,但是实际情况是它会随细胞膜电位和神经元内部状态的变化而变化。
还有,反转电位 $E_\text{GABA}$ 被假设为常数,但是在高频率刺激下,细胞可能无法保持在平衡状态 \cite{Staley1995},因此导致 $E_\text{GABA}$ 偏移。

在估计模型参数的时候,突触后的递质接收器的动力学被假设为单指数递减型,这个限制了分析方法应用于某些神经元突触。
首先突触的饱和作用就被忽略了。
虽然对于更复杂的突触类型,释放率统计会变得更困难,但是突触释放模型的核心不受影响,因为它只模拟释放过程,而不受递质接收器的动力学影响。

综上,一些假设限制了估计的释放率的准确度,进而影响参数拟合的准确度。
虽然量化分析的结果可能会有些变化,但应该不会影响非同步释放的定性描述。

\section{单突触仿真结果讨论}
\label{section:discussion:synapse-simulation}
单突触仿真的对非同步释放强度的调控与实验现象对应,因此模型揭示了不同实验调控方法提高非同步释放强度的一些差异(见第\ref{section:result:modulate-with-tau-ar}节和第\ref{section:result:modulate-with-u-ar}节)。

此外第\ref{section:result:modulate-x-0}节的结果显示了实验中尚未研究过的现象。
量子相对大小$wx_0$的变化会改变 IPSC 的平滑度。
如果采用 IPSC 斜率来探测非同步释放事件,可能会加大阈值挑选的难度。
根据模型,如果要用 IPSC 的斜率来探测非同步释放事件,那么应该先推断出量子大小 $wx_0$,然后根据量子大小设置阈值。

\section{网络仿真结果讨论}
\label{section:discussion:network-simulation}
前人实验认为非同步释放会使网络发放时间精确性下降 \cite{Manseau2010},进而使网络去同步。
仿真计算发现非同步释放造成的发放精度下降依赖于突触连接密度,连接较密集时影响较小。
实验中没有观察到这种现象,可能是因为实验材料中回路较小,突触连接稀疏。

此外由于竞争囊泡资源,非同步释放的增强会导致同步释放的减弱。
而同步释放使网络同步的作用比非同步释放强,即每个周期内不同神经元的相位差改变大。
因此,非同步释放增强可能使网络同步行为对噪声的抵抗力降低。
所以可以认为非同步释放增强,可以通过影响同步释放间接影响网络同步性。

此外,以前一些实验中用电导 $g$ 来衡量抑制性水平 \cite{Wang1996,Voegtlin2007},而网络仿真的结果显示,相同的平均电导 $E(g)$ 也可能有不同的抑制性水平(见第\ref{section:result:network-self-inhibition}节)。选择发放率作为抑制性指标能够揭示被忽视的作用机制。

\section{总结}
我们展示了包含同步释放和非同步释放的化学突触模型。
模型基于大型网络仿真计算中常用的短时程突触可塑性模型,计算效率较高,适合未来的模型研究。
本文进行了简单的网络模拟,得出了一些结果。
使得对以往实验的理解又深入了一步。
希望这一模型能够促进非同步释放研究,增加对它在神经系统中的计算特性和神经功能的理解。



%%% 其它部分
\backmatter

% 本科生要这几个索引,研究生不要。选择性留下。
\makeatletter
\ifthu@bachelor
  % 插图索引
  %\listoffigures
  % 表格索引
  %\listoftables
  % 公式索引
  %\listofequations
\fi
\makeatother

% 参考文献
\bibliographystyle{thubib}
\bibliography{ref/refs}

% 学术成果
\begin{resume}
  \begin{enumerate}[{[}1{]}]
  \item Wang T\textsuperscript{1}, Yin L, Rasch MJ, Shu Y, Wu S\textsuperscript{*}. A Phenomenological Synapse Model for Asynchronous Neurotransmitter Release. {\it PLoS Computational Biology} (SCI 2区, 审稿中)
  \end{enumerate}

\end{resume}


% 致谢
\include{data/acknowledgements}

% 附录
%\begin{appendix}
%\input{data/appendix01}
%\end{appendix}


\end{document}
